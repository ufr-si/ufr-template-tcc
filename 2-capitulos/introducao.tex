
% ----------------------------------------------------------
% Introdução (exemplo de capítulo sem numeração, mas presente no Sumário)
% ----------------------------------------------------------
\chapter{Introdução}

A introdução é o primeiro elemento textual e contém alguns itens importantes do projeto de pesquisa: tema, questões de pesquisa, objetivos e justificativa (sucinta). Deve situar o autor da pesquisa em relação ao que irá estudar, apresentando em linhas gerais como chegou ao tema e como pretende desenvolvê-lo em sua pesquisa. Ela deve se encerrar apresentando ao leitor a organização retórica de seu trabalho, ou seja, as partes que compõem o TCC.

\section{Olá mundo}

Segundo Machado, Lousada e Abreu-Tardelli (2005, p. 83), “a introdução pode ser vista como um trailer do que o leitor verá no seu trabalho, nem mais nem menos”. É uma seção que deve levar o leitor a querer ler o trabalho, seduzindo-o.

\subsection{Olá mundo}

Uma dica útil dada pelas autoras é apresentar inicialmente o “objeto” sobre o qual trata a pesquisa em um relato de como você chegou ao tema, quais os motivos mais relevantes, as buscas que efetuou, as decisões tomadas e as teorias que foi selecionando ao longo dessa busca. 

\subsubsection{Olá mundo}

Aqui serão dadas indicações gerais para a apresentação gráfica de seu trabalho, contudo, você pode consultar a NBR 14724:2011 para obter mais informações sobre a apresentação de trabalhos acadêmicos.

\subsubsubsection{Olá mundo}

Todo o texto deve ser digitado em espaço 1,5 cm, exceto: citações de mais de três linhas, notas de rodapé, referências, legendas e fontes das ilustrações e das tabelas que devem ser digitados em espaço simples. As referências, ao final do trabalho, devem ser separadas entre si por um espaço simples em branco.

\begin{citacao}
    Esta é uma citação, \lipsum[20]
\end{citacao}

As margens da página devem ser de 3 cm nas margens esquerda e superior e 2 cm nas margens direita e inferior.

Os títulos dos capítulos (seções primárias, secundárias, etc.) devem ser digitados após a sua numeração (indicação de seção), separados por um espaço. O texto deve iniciar em outra linha, separado por um espaço entrelinhas de 1,5. 

As seções primárias iniciam-se em nova página e são grafadas em caixa alta e negrito. As seções secundárias são grafadas em negrito com apenas a primeira letra maiúscula. As seções terciárias não são grafadas em negrito. Escreva um título criativo consoante o arcabouço teórico e seu plano de trabalho constante no sumário. Utilize junto a este template a NBR 6024:2012 – Numeração progressiva das seções de um documento.

A fonte utilizada no texto é Arial ou Times, tamanho 12, excetuando-se citações com mais de três linhas, notas de rodapé, paginação, legendas e fontes das ilustrações e das tabelas, que devem ser em tamanho 11. 

As citações diretas com mais de três linhas devem ser destacadas com recuo de 4 cm da margem esquerda, fonte tamanho 11 e sem as aspas. Consulte sempre a norma específica para citações, NBR 10520:2002.

As páginas pré-textuais (todas que precedem a Introdução) devem ser contadas, mas não numeradas, exceto a capa e página da Ficha Catalográfica. A numeração deve figurar a partir da Introdução, em algarismos arábicos, no canto superior direito da folha, a 2 cm da borda superior, ficando o último algarismo a 2 cm da borda direita da folha, fonte 11.


% \input{tabelas/tabela1.tex}