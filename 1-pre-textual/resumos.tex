% ---
% RESUMOS
% ---
% resumo em português

\begin{resumo}
 De acordo com a ABNT NBR 6028:2021, o resumo informativo deve ressaltar o objetivo, o método, os resultados e as conclusões do documento. Ele deve ser composto de uma sequência de frases concisas e afirmativas. Convém usar o verbo na terceira pessoa do singular. O texto do resumo deve ser digitado em um parágrafo único, justificado. O espaçamento entre linhas é simples e o tamanho da fonte é 12. Deve conter de 150 a 500 palavras. As palavras-chave devem figurar logo abaixo do resumo, antecedidas da expressão Palavras-chave, seguida de dois-pontos, separadas entre si por ponto e vírgula e finalizadas por ponto. Devem ser grafadas com as iniciais em letra minúscula, com exceção dos substantivos próprios e nomes científicos.
 
\textbf{Palavras-chave}: latex; abntex; editoração de texto.

\end{resumo}


% resumo em inglês
\begin{resumo}[Abstract]
 \begin{otherlanguage*}{english}
   \lipsum[150]

   \vspace{\onelineskip}
 
   \noindent 
   \textbf{Keywords}: latex; abntex; text editoration.
 \end{otherlanguage*}
\end{resumo}

% Se quiser, é possível adicionar outros idiomas para resumo, bastando utilizar os exemplos abaixo.  

% % resumo em francês 
% \begin{resumo}[Résumé]
% \SingleSpacing
%  \begin{otherlanguage*}{french}
%     Il s'agit d'un résumé en français.
 
%    \textbf{Mots-clés}: latex. abntex. publication de textes.
%  \end{otherlanguage*}
% \end{resumo}

% % resumo em espanhol
% \SingleSpacing
% \begin{resumo}[Resumen]
%  \begin{otherlanguage*}{spanish}
%    Este es el resumen en español.
  
%    \textbf{Palabras clave}: latex. abntex. publicación de textos.
%  \end{otherlanguage*}
% \end{resumo}
% ---